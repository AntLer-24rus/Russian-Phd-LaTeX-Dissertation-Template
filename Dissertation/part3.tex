\chapter{Анализ данных эксперимента}

После проведения экспериментальных исследований влияния радиуса закругления рабочей кромки и шага резания на составляющие силы, возникающей на дисковом инструменте, при механическом разрушении льда получен набор <<сырых>> данных, который включает в себя:
\begin{itemize}
	\item фотографии осколков;
	\item графики переходных процессов для каждого сочетания факторов.
\end{itemize}

Дальнейшее их использование предполагает обработку и оценку корректности методами математики и статистики, такими как:
\begin{itemize}
%	\item оценка дисперсии;
	\item усреднение значений;
	\item фильтрация постоянной составляющей;
	\item отброс грубых ошибок;
	\item сглаживание.
\end{itemize}

Далее в это главе процесс обработки и оценки адекватности данных будет приведен последователь.

\section{Фильтрация и сглаживание переходных процессов резания льда}
Описание методов регрессионного анализа очень кратко. Описание работы программного комплекса по работе с изображением, описание программного комплекса по работе с данными

\section{Усреднение повторных опытов}

Из параграфа \ref{sect2_2} известно что для каждый опыт повторялся 5 раз, для учета неизвестных факторов, а это значит что данные необходимо усреднить.

Точечная оценка величины \todo{какой}:
\begin{equation}\label{eq:x_ocenka}
\hat{a}=\frac{1}{т}\sum_{i=1}^{n} x_i,
\end{equation}
где $ n $ "--- число повторных опытов, $ x_i $ "--- измеренное значение в отдельном опыте \cite{Zajigaev}.

Среднеквадратичное отклонение точечной оценки 
Оценить точность позволит относительная погрешность измерений, вычисляемая по следующее формуле:
\begin{equation}\label{eq:Error}
\varepsilon=\frac{1}{n}\sum_{i=1}^{n} \frac{\left| x_i-\bar{x}\right| }{\bar{x}}\cdot100\%
\end{equation}