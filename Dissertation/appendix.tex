\input{Dissertation/appendixsetup}   % Предварительные настройки для правильного подключения Приложений
\chapter{Функции обработки данных эксперимента} \label{Appendix:FPE}

В данном приложении приведены все функции для предварительной обработки эксперементальных данных.

\section{Функция удаления постоянной составляющей}\label{Appendix:FPE:COF}

\noindent\textbf{Входные параметры:}
\begin{itemize}[leftmargin=1.25cm]
	\item [] \lstinline{x} "--- массив измеренных значений (сигнал переходного процесса).
\end{itemize}
\textbf{Выходные параметры:}
\begin{itemize}[leftmargin=1.25cm]
	\item [] \lstinline{ix} "--- массив измеренных значений за вычетом постоянной составляющей.
\end{itemize}

\begin{lstlisting}[label={list:clearOffSet},caption={Функция удаления постоянной составляющей}]
function ix = clearOffSet(x)
	N = size(x,2);
		if mod(N, 2) ~= 0
			x(end) = [];
		end
	X = fftshift(fft(x));
	X(end/2+1) = 0;
	ix = ifft(ifftshift(X));
\end{lstlisting}

\section{Функция отброса грубых ошибок}\label{Appendix:FPE:DGE}

\noindent\textbf{Входные параметры:}
\begin{itemize}[leftmargin=1.25cm]
	\item [] \lstinline{y_in} "--- массив измеренных значений (сигнал переходного процесса);
	\item [] \lstinline{f} "--- выбор режима работы: 1 "--- анализировать точки попавшие во вторую группу, 0 "--- анализировать только точки попавшие в третью группу.
\end{itemize}
\textbf{Выходные параметры:}
\begin{itemize}[leftmargin=1.25cm]
	\item [] \lstinline{out} "--- индексный массив с 0 и 1 указывающий индексы отбрасываемых точек.
\end{itemize}

\begin{lstlisting}[label={list:DropGrossError},caption={Функция отброса грубых ошибок}]
	function out = DropGrossError(y_in, f)
		p_l = 5;    l=1;
		p_r = 0.1;  r=2;
		
		n = length(y_in);
		out = zeros(1, n);
		y_error = false(1, n);
		y_warn = false(1, n);
		y_good = false(1, n);
		y_idx = 1:n;
		e = true;
		while e
			y_idx_step = y_idx(~y_error);
			y_step = y_in(~y_error);
			
			n = length(y_step);
			t_tab = abs(...
				(tinv([p_l p_r] ./ 100, n - 2) .* sqrt(n-1)) ./ ...
				sqrt(n - 2 + tinv([p_l p_r] ./ 100, n - 2) .^ 2) ...
				);
			
			y_t = zeros(1, 2);
			ind_y_t = zeros(1, 2);
			
			[y_t(1), ind_y_t(1)] = max(y_step);
			[y_t(2), ind_y_t(2)] = min(y_step);
			
			[d_max, d_max_ind] = max(abs(y_t-mean(y_step)));    
			d_max_ind = ind_y_t(d_max_ind);
			
			t = d_max/std(y_step);
			
			e = t > t_tab(r);
			w = t_tab(l) < t & t < t_tab(r);
			g = t < t_tab(l);
			
			if w && f
				hFq = figure;
				plot(y_step);hold on
				plot(d_max_ind, y_step(d_max_ind), 'sr', ...
				'MarkerFaceColor','r')
				axis([0 n min(y_step)-2 max(y_step)+2])
				choice = ...
				    questdlg('Удалить отмеченну на графике точку?', ...
				             'Удаление грубых точек', ...
				             'Да','Нет','Нет');
				close(hFq)   
				e = strcmp('Да', choice);
			end
			
			y_error(y_idx_step(d_max_ind)) = e;
			y_warn(y_idx_step(d_max_ind)) = w;
			y_good(y_idx_step(d_max_ind)) = g;    
		end
		out(y_error) = 1;		
\end{lstlisting}