\chapter*{Список сокращений и условных обозначений}             % Заголовок
\addcontentsline{toc}{chapter}{Список сокращений и условных обозначений}  % Добавляем его в оглавление
\noindent
\addtocounter{table}{-1}% Нужно откатить на единицу счетчик номеров таблиц, так как следующая таблица сделана для удобства представления информации по ГОСТ
%\begin{longtabu} to \dimexpr \textwidth-5\tabcolsep {r X}
\begin{longtabu} to \textwidth {r X}
	\textbf{СЛО}	& снежно-ледяные образования\\
	\textbf{ПСЛО}	& прочные снежно-ледяные образования\\
	\textbf{СКО}	& среднеквадратичное отклонение (выборочное)\\
	\textbf{БПФ}	& быстрое преобразование Фурье\\
	\textbf{АЦП}	& аналого цифровой преобразователь\\
	\textbf{ППП}	& пакет прикладных программ
\end{longtabu}
