%Актуальность работы
{\actuality} Общая протяжённость российской сети автодорог общего пользования федерального, регионального и местного значения оценивается Росавтодором в 1~396~000~км, в том числе 50~800~км федерального значения. Постоянно возрастающая величина грузонапряжённости, интенсивности и скорости движения влечёт за собой увеличение трудозатрат на их содержание. Наиболее затратным является зимнее содержание автодорог, так как 70~\% из них расположено в зонах, где длительность зимнего периода превышает 140 дней в году.

Согласно стратегии \cite{Strategi} развития Арктической зоны Российской Федерации и обеспечения национальной безопасности на период до 2020 года, утвержденной 8~февраля~2013 года предусмотрена интеграция Арктической зоны с основными районами России по средствам: формирования опорной сети автомобильных дорого и современных транспортно-логистических узлов; развития, реконструкции и модернизации аэропортовой сети. Содержание дорог различного назначения, аэродромов и вертолетных площадок в Арктической зоне потребует разработку и внедрение современных высокоэффективных рабочих органов дорожных машин для зимнего содержания, адаптированных к использованию в арктических условиях.

Среди основных задач зимнего содержания автодорог можно выделить механический метод удаление снежно-ледяных образований (СЛО) с помощью отвальных (плужных), щёточных, шнекороторных, фрезерно-роторных и других рабочих органов дорожных машин. Однако, в случае формирования прочных СЛО качественная очистка рабочими органами, перечисленными выше, затрудняется или становится невозможной.

Для повышения эффективности и снижения энергоемкости при удалении прочных СЛО также предложено применение дискового режущего инструмента \todo{[литература/патенты]}. С применением дискового режущего инструмента встает вопрос создания высокоэффективных рабочих органов, для проектирование которых необходимо заранее знать нагрузочные параметры, величина которых зависит от множества факторов. Например, таких как: скорость резания, геометрические параметры инструмента, температура окружающей среды, степени износа режущей кромки.

Дисковый режущий инструмент получил широкое освещение в области горнодобывающей промышленности, а именно широко применяется в проходческих комбайнах при разработке горных пород. Существует множество работ \todo{[литература]}, рассматривающих влияние различных факторов на силу сопротивления резанию. Однако, в этих работах рассматривается резание грунтов и горных пород и не уделено внимание прочным СЛО и льду (как частному случаю прочных СЛО). В работах \todo{[Ганжа, Ковалевич]} рассматривается влияние некоторых факторов на силу сопротивления прочных СЛО резанию. Влияние же степени износа режущей кромки на силу сопротивления прочных СЛО резанию изучено недостаточно. Поэтому поиск новых методов расчета и обоснование рабочих параметров, учитывающих степень износа режущей кромки дискового инструмента является актуальной задачей.


% {\progress} 
% Этот раздел должен быть отдельным структурным элементом по
% ГОСТ, но он, как правило, включается в описание актуальности
% темы. Нужен он отдельным структурынм элемементом или нет ---
% смотрите другие диссертации вашего совета, скорее всего не нужен.

% Цель работы
{\aim} данной работы является обоснование рационального, с точки зрения минимизации энергозатрат и повышения производительности, радиуса закругления рабочей кромки дискового режущего инструмента, \todo{а также разработка конструкции сменного рабочего органа с дисковым режущим инструментом для разрушения прочных СЛО.} 

% Задачи работы
Для~достижения поставленной цели необходимо было решить следующие {\tasks}:
\begin{enumerate}
  \item Разработать метод и комплекс средств контроля силы сопротивления резанию дисковым инструментом прочных СЛО, учитывающий влияние радиуса закругления рабочей кромки и шага резания.
  \item Исследовать влияния радиуса закругления рабочей кромки и шага резания на силы, возникающие на дисковом режущем инструменте при механическом разрушении прочных СЛО.
  \item \todo{Разработать математическую модель процесса взаимодействия дискового режущего инструмента со снежно-ледяными образованиями, учитывающую влияние радиуса закругления рабочей кромки и шага резания.}
  \item Разработать методику обоснования рационального радиуса закругления рабочей кромки дискового режущего инструмента входящего в состав сменного рабочего органа дорожных машин.
\end{enumerate}

%Научная новизна
{\novelty}
\begin{enumerate}
  \item Впервые получен метод контроля силы сопротивления резанию дисковым инструментом при разрушения прочных СЛО, включающий комплексную оценку влияния радиуса закругления рабочей кромки и шага.
  \item Было выполнено оригинальное исследование силы сопротивления льда резанию дисковым инструментом с различным радиусом закругления рабочей кромки.
  \item Впервые получена математическая модель процесса взаимодействия дискового режущего инструмента со снежно-ледяными образованиями, позволяющая определять составляющие горизонтальной, боковой и вертикальной сил резания, возникающих на дисковом режущем инструменте и учитывающая влияние радиуса закругления рабочей кромки и шага резания.
  \item Впервые получена методика обоснования рационального радиуса закругления рабочей кромки дискового режущего инструмента, входящего в состав сменных рабочих органов дорожных машин, позволяющая: увеличить производительность; снизить энергоемкости \todo{и обеспечить сохранность дорожного полотна.}
\end{enumerate}

% Практическая значимость
{\influence} Разработанные математическая модель и методика, позволяют оценивать влияние радиуса закругления рабочей кромки и шага резания на силу сопротивления резанию, определять нагрузочные параметры и энергоэффективность процесса разрушения прочных СЛО.

% Методология исследования
{\methods} Решение поставленных задач осуществлялось с использованием комплексного подхода, включающего анализ существующего опыта по созданию методов контроля нагрузочных параметров при разрушении мерзлых и не мерзлых грунтов, горных пород и снежно-ледяных образований различным режущим инструментом. Экспериментальные лабораторные исследования процесса резания льда проводились полноразмерным дисковым режущим инструментом. При выполнении работы применялись: поверенные стандартные и специально разработанные автором приборы; теория планирования и обработки результатов экспериментальных исследований; методы математической статистики и регрессионного анализа.

% Положения выносимые на защиту
{\defpositions}
\begin{enumerate}
  \item Метод и комплекс средств контроля силы сопротивления резанию дисковым инструментом при разрушения прочных СЛО, включающий комплексную оценку влияния радиуса закругления рабочей кромки и шага.
  \item Результаты экспериментальных исследований влияния радиуса закругления рабочей кромки дискового режущего инструмента и шага резания на силу сопротивления резанию при механическом разрушении прочных СЛО.
  \item \todo{Математическая модель процесса взаимодействия дискового режущего инструмента с прочными СЛО, учитывающая влияние радиуса закругления рабочей кромки дискового режущего инструмента и шага резания.}
  \item Методика обоснования рационального радиуса закругления рабочей кромки дискового режущего инструмента, входящего в состав сменных рабочих органов дорожных машин.
\end{enumerate}
%В папке Documents можно ознакомиться в решением совета из Томского ГУ
%в файле \verb+Def_positions.pdf+, где обоснованно даются рекомендации
%по формулировкам защищаемых положений. 

{\reliability} полученных результатов обеспечивается достаточным объемом проведенных экспериментальных исследований, использованием средств контроля прошедших поверку, сходимостью теоретических и экспериментальных данных. Результаты находятся в соответствии с результатами, полученными другими авторами.


{\probation}
Основные результаты работы докладывались~на:
перечисление основных конференций, симпозиумов и~т.\:п.

{\contribution} Автор принимал активное участие в разработке комплекса средств контроля нагрузочных параметров дискового режущего инструмента для разрушения прочных СЛО. Автором лично разработан метод контроля силы сопротивления учитывающий влияние радиуса закругления рабочей кромки дискового режущего инструмента и шага резания. Проведены исследования, согласно с разработанным методом, и их статистическая и математическая обработка. \todo{Разработан программный комплекс автоматической обработки результатов исследований.} 

%\publications\ Основные результаты по теме диссертации изложены в ХХ печатных изданиях~\cite{Sokolov,Gaidaenko,Lermontov,Management},
%Х из которых изданы в журналах, рекомендованных ВАК~\cite{Sokolov,Gaidaenko}, 
%ХХ --- в тезисах докладов~\cite{Lermontov,Management}.

\ifnumequal{\value{bibliosel}}{0}{% Встроенная реализация с загрузкой файла через движок bibtex8
    \publications\ Основные результаты по теме диссертации изложены в XX печатных изданиях, 
    X из которых изданы в журналах, рекомендованных ВАК, 
    X "--- в тезисах докладов.%
}{% Реализация пакетом biblatex через движок biber
%Сделана отдельная секция, чтобы не отображались в списке цитированных материалов
    \begin{refsection}[vak,papers,conf]% Подсчет и нумерация авторских работ. Засчитываются только те, которые были прописаны внутри \nocite{}.
        %Чтобы сменить порядок разделов в сгрупированном списке литературы необходимо перетасовать следующие три строчки, а также команды в разделе \newcommand*{\insertbiblioauthorgrouped} в файле biblio/biblatex.tex
        \printbibliography[heading=countauthorvak, env=countauthorvak, keyword=biblioauthorvak, section=1]%
        \printbibliography[heading=countauthorconf, env=countauthorconf, keyword=biblioauthorconf, section=1]%
        \printbibliography[heading=countauthornotvak, env=countauthornotvak, keyword=biblioauthornotvak, section=1]%
        \printbibliography[heading=countauthor, env=countauthor, keyword=biblioauthor, section=1]%
        \nocite{%Порядок перечисления в этом блоке определяет порядок вывода в списке публикаций автора
                vakbib1,vakbib2,%
                confbib1,confbib2,%
                bib1,bib2,%
        }%
        \publications\ Основные результаты по теме диссертации изложены в \arabic{citeauthor} печатных изданиях, 
        \arabic{citeauthorvak} из которых изданы в журналах, рекомендованных ВАК, 
        \arabic{citeauthorconf} "--- в тезисах докладов.
    \end{refsection}
%    \begin{refsection}[vak,papers,conf]%Блок, позволяющий отобрать из всех работ автора наиболее значимые, и только их вывести в автореферате, но считать в блоке выше общее число работ
%        \printbibliography[heading=countauthorvak, env=countauthorvak, keyword=biblioauthorvak, section=2]%
%        \printbibliography[heading=countauthornotvak, env=countauthornotvak, keyword=biblioauthornotvak, section=2]%
%        \printbibliography[heading=countauthorconf, env=countauthorconf, keyword=biblioauthorconf, section=2]%
%        \printbibliography[heading=countauthor, env=countauthor, keyword=biblioauthor, section=2]%
%        \nocite{vakbib2}%vak
%        \nocite{bib1}%notvak
%        \nocite{confbib1}%conf
%    \end{refsection}
}
%При использовании пакета \verb!biblatex! для автоматического подсчёта
%количества публикаций автора по теме диссертации, необходимо
%их здесь перечислить с использованием команды \verb!\nocite!.
    

